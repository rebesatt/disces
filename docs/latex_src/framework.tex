\section{The \sys{} Framework}
\label{sec:framework}
This section introduces \sys{} as a framework for the systematic design of
algorithms to address the problem of event query discovery. We first discuss
dimensions along which design choices are captured (\autoref{sec:dimensions}),
before turning to their combination (\autoref{sec:combinations}).
\begin{table*}
    \smaller
	     \caption{Illustration of the dimensions that are incorporated in
	     the design of
		discovery algorithms as part of the \sys{} framework.}
	\label{fig:dimensions_overview}
	\vspace{-1em}
	\begin{tabular}{crclll|llll|llll|lll}
        \toprule
        \multicolumn{6}{c}{Direction}
                                & \multicolumn{4}{c}{Strategy} &
        \multicolumn{4}{c}{Construction: Separated | Unified} & \multicolumn{3}{c}{Attributes: Separated | Comprehensive} \\
        \midrule
        \multicolumn{5}{c}{$\langle(a,5,11)\rangle$}             &                                                                                                                        &  \multicolumn{1}{l}{\tcbox[colback=white, colframe=black, on line, size=fbox]{\underline{$\langle(a,5)\rangle$}} }  &  &  &  & \multicolumn{1}{c}{{$\langle(x,y),(x,y)\rangle$}} & \multicolumn{1}{c}{{$\langle(a,5)\rangle$}} &   \multicolumn{1}{c}{{$\langle(x,5),(x,\_)\rangle$} }    &      &  \multicolumn{1}{c}{$\langle(x),(a),(x)\rangle$}    & \multicolumn{1}{c}{$\langle(5),(7)\rangle$} &  \multicolumn{1}{c}{$\langle(x,5),(x,7)\rangle$}  \\
        \multicolumn{1}{l}{bottom} & \multirow{2}{*}{$\uparrow$} & \multirow{2}{*}{$\vdots$} & \multirow{2}{*}{$\downarrow$} & top & & \multicolumn{1}{c}
     {$\uparrow$}   &  & & &  \multicolumn{1}{c}{$\uparrow$} &
     \multicolumn{1}{c}{$\uparrow$} & \multicolumn{1}{c}{$\uparrow$}  &  &
     \multicolumn{1}{c}{$\color{black}\uparrow$}  &
     \multicolumn{1}{c}{$\uparrow$} &      \multicolumn{1}{c}{$\uparrow$}
     \\
        \multicolumn{1}{l}{up}     &
        &                           &                               & down
        &  &  \multicolumn{3}{c}{\tcbox[colback=white, colframe=black, on
        line, size=fbox]{\underline{$\langle(a,\_)\rangle$};
        $\langle(\_,5)\rangle$}}             &              &
        \multicolumn{1}{c}{{$\langle(x,\_),(x,\_)\rangle$}} &
        \multicolumn{1}{c}{$\langle(a,\_)\rangle$}&
        \multicolumn{1}{c}{$\langle(x,\_),(x,\_)\rangle$}       &   &
        \multicolumn{1}{c}{$\langle(x),(x)\rangle$}    &
        \multicolumn{1}{c}{$\langle(5)\rangle$} &
        \multicolumn{1}{c}{$\langle(x,5),(x,\_)\rangle$}
        \\[1em]
        \multicolumn{5}{c}{$\langle(a,\_,\_)\rangle$} &
                              & \multicolumn{3}{c}{\underline{DFS} |
        \tcbox[colback=white, colframe=black, on line, size=fbox]{BFS}}  & &
        \multicolumn{1}{c}{\textcolor{black}{pattern queries} } &
        \multicolumn{1}{c}{\textcolor{black}{type queries}}     &
        \multicolumn{1}{c}{\textcolor{black}{mixed queries}}          & &
        \multicolumn{1}{c}{\textcolor{black}{attribute 1}}    &
        \multicolumn{1}{c}{\textcolor{black}{attribute 2}} & all attributes
        \\
        \bottomrule
       \end{tabular}
	\vspace{-1em}
\end{table*}
\subsection{Dimensions of Design Choices}
\label{sec:dimensions}
The \sys{} framework includes four dimensions for design choices that guide how
the space of candidate queries is explored. These dimensions, illustrated in
\autoref{fig:dimensions_overview}, refer to the following properties of the search
through that space, which are detailed in the remainder:
\begin{description}[itemsep=.3em,font=\normalfont\itshape,
labelindent=0em,leftmargin=2.5em]
    \item[Direction:] The search proceeds bottom-up or top-down.
    \item[Strategy:] The approach is depth-first or
    breadth-first.
    \item[Construction:] Type and pattern queries are initially
    constructed separately, or mixed queries are immediately
    considered.
    \item[Attributes:] Attributes are initially considered
    separately, or immediately incorporated comprehensively.
\end{description}
\textbf{Direction.}
When traversing the space of candidate queries, two directions may be considered:
Bottom-up approaches start with the most generic query possible and, by adding
attribute values or variables, generate stricter queries.
The traversal stops, once the sequences in the database no longer support the
explored queries.
Top-down approaches start with a most specific query,
i.e., a shortest \change{stream} of the given database.
Then, they explore the search space by deleting attribute values or exchanging
them with variables; stopping whenever queries that are supported by all sequences
of the database have been found. Considering the example in
\autoref{fig:dimensions_overview}, a bottom-up approach evaluates the query
$\langle (a,\_,\_) \rangle$ before the more specific query $\langle (a,5,11)
\rangle$, and vice versa for a top-down approach.
\textbf{Strategy.} To explore candidate queries, one may adopt a
depth-first search (DFS) strategy. Then, the space is traversed by
generalizing or specializing the queries until a query with a different
support behavior is reached, i.e., a non-supported query is generalized
until it is supported (top-down direction) or a supported query is
specialized until it is no longer supported (bottom-up direction). A
different strategy
is to adopt breadth-first search (BFS), i.e., to first explore all
queries containing an equal number of non-placeholders, before continuing with
those with less (top-down
direction) or more (bottom-up
direction) non-placeholders. In \autoref{fig:dimensions_overview},  assuming a
bottom-up
search direction, from query
$\langle (a,\_) \rangle$, DFS would continue with the stricter query
$\langle (a,5) \rangle$, whereas BFS would first consider queries with the
same number of non-placeholders, such as $\langle (\_,5) \rangle$.
\textbf{Construction.}
Another algorithmic choice is whether to construct type queries and pattern
queries separately or with a unified approach. The former means that the
space of candidate type queries and the space of candidate pattern
queries are explored independently, before merging the results to also
obtain the descriptive mixed queries. Note that the isolated discovery of
type queries corresponds to the common problem of maximal frequent sequence
mining~\cite{10.1007/978-3-319-06483-3_8,10.1007/978-3-642-53914-5_15,10.1145/2757217}.
In a unified approach, in turn, all query structures are explored as part of
a single search space. In \autoref{fig:dimensions_overview}, a separated
approach would explore pattern queries, e.g., $\langle (x,\_),(x,\_) \rangle$
and subsequently $\langle (x,y),(x,y) \rangle$, and type queries, e.g.,
$\langle (a,\_)\rangle$ and then $\langle(a,5) \rangle$ before merging them. A
unified approach would directly construct mixed queries and explore, for
instance, $\langle (x,5),(x,\_)\rangle$ after $\langle (x,\_),(x,\_)\rangle$.
\textbf{Attributes.}
Similarly, the attributes of an event schema induce a
design choice for the exploration of the space of candidate
queries. We may first
explore each attribute separately, before merging the
results to obtain the final set of descriptive queries; or rely on a
comprehensive approach that explores all attributes
simultaneously. Again, \autoref{fig:dimensions_overview} illustrates this
design choice: We may consider the first attribute, exploring $\langle (x),(x)
\rangle$ followed by $\langle (x),(a),(x) \rangle$, and the second attribute,
exploring $\langle (5) \rangle$ followed by $\langle (5),(7) \rangle$ and merging them afterward;
or immediately consider both attributes, by exploring $\langle
(x,5),(x,\_) \rangle$ and then $\langle (x,5),(x,7) \rangle$.
\subsection{Combination of Design Choices}
\label{sec:combinations}
Having described fundamental choices in the design of discovery algorithms,
we review their interplay and underlying assumptions.
First, the choice regarding the search direction, top-down
vs. bottom-up, relates to an important assumption on the application scenario. A
top-down search strategy will commence with a shortest stream of the
database as a query, and step-wise generalize it until queries supported by
the whole database are found. Hence, such an approach can be expected
to work efficiently, if descriptive queries are only slightly shorter than
the shortest stream in the database. In the scenarios outlined
in \autoref{sec:motivation}, however, queries commonly
contain solely a few terms and are generally much shorter than the available
streams. Hence, a top-down exploration of candidate queries will
quickly become intractable, due to the sheer size of the
respective search space. In the remainder, we therefore focus on the
instantiation of algorithms that adopt a bottom-up direction for
the search.
Second, we focus on the combination of search strategies (DFS vs. BFS) and
the construction approach (type/pattern-separated vs. unified). As mentioned
above, the separate construction of type queries and pattern queries enables
us to incorporate existing results: The discovery of type queries in
isolation corresponds to the maximal frequent sequence mining (MFSM)
problem~\cite{agrawal1995}. Since state-of-the-art algorithms for the MFSM
problem rely on BFS,
we adopt it as the search strategy in any approach that is based on the
separate construction of type queries and pattern queries.
In contrast, for the unified construction of queries, a BFS search strategy is
harmful. A bottom-up, unified
construction of queries following BFS will explore mixed queries, for
which it is known that they cannot be descriptive. For instance, the mixed
queries
$\langle (a),(a),(x_0),(x_0)\rangle$ and $\langle (x_0),(x_0),(b),(b) \rangle$ will be
considered as candidates if the queries $\langle (a),(a),(b),(b) \rangle$ and
$\langle (x_0),(x_0),(x_1),(x_1) \rangle$ are supported by the
stream database, even
though the mixed queries cannot be descriptive. A separated construction of
queries avoids the issue, as
does the combination of a unified construction with DFS.
Third, the question whether to consider the attributes
separately or comprehensively is largely orthogonal to the above
design choices. Hence, the separate or comprehensive treatment of attributes
may be combined with either of the above design choices.
Based thereon, we derive the four combinations of design choices
listed in \autoref{tab:algos}, which can be deemed suitable to design
efficient algorithms for the problem of event query discovery.
\begin{table}[h!]
\vspace{-.5em}
\caption{Combination of design choices for discovery algorithms.}
\label{tab:algos}
\vspace{-1em}
\footnotesize
\begin{tabular}{l@{\hspace{1em}} l@{\hspace{1em}} l@{\hspace{1em}}
l@{\hspace{1em}} l}
		\toprule
		& Direction & Strategy & Construction & Attributes \\
		\midrule
		B-S-S & bottom-up & \underline{B}FS & \underline{S}eparated &
		\underline{S}eparated\\
		B-S-C & bottom-up & \underline{B}FS & \underline{S}eparated &
		\underline{C}omprehensive \\
		D-U-S & bottom-up & \underline{D}FS & \underline{U}nified &
		\underline{S}eparated \\
		D-U-C & bottom-up & \underline{D}FS & \underline{U}nified &
		\underline{C}omprehensive \\
		\bottomrule
\end{tabular}
\end{table}