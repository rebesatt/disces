\section{Related Work}
\label{sec:related_work}
Closest to our work are iCEP~\cite{icep} and the IL-Miner~\cite{ilminer},
which both address the problem of event query discovery based on historic
streams. Yet, both
algorithms take ad-hoc design choices and adopt a restricted query model.
iCEP cannot discover queries, in which attribute values
occur multiple times. This limitation is overcome by the IL-Miner, which,
however, cannot discover queries with query terms containing only
variables. We showed empirically that this lossy version of the IL-Miner
suffers from incomplete results. Also, our algorithms outperform an extended
version of the IL-Miner and extend the set of problem
instances that may be addressed.
Our query model and the notion of descriptiveness is inspired
by~\cite{icdt2022,DBLP:conf/btw/Kleest-Meissner23}. Yet, the discovery
algorithms proposed in~\cite{icdt2022,DBLP:conf/btw/Kleest-Meissner23} are
limited to finding
\emph{some} descriptive queries, not all of them.
Machine learning approaches to anticipate situations of
interest provide an alternative angle to stream-based
monitoring~\cite{yanli2021,ARECEP,autoCEP}. The main drawback of these
methods,
however, is the lack of traceability of the derived
predictions. Using an approach to construct probabilistic state
machines based on representation learning~\cite{yanli2021} in our
evaluation, we observed correctness and completeness issues.
Event query discovery is linked to frequent sequence
mining~\cite{agrawal1995,clospan,bide}, which considers solely
the level of attribute values, though. Our query model embeds a finite
sequence into
another sequence that satisfies certain
constraints. This problem setting is ubiquitous in foundational algorithmic
research, especially in combinatorial string
matching~\cite{DayEtAl2022,KoscheEtAl2022}.
Pattern discovery has also been investigated for
time-series data~\cite{streaming,crossmatch}, which, again, works only  on
attribute values and ignores criteria to correlate events by means of
variables.